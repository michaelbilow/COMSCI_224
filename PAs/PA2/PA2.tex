\documentclass{article}
\usepackage{hyperref}
\usepackage{rotating}
\usepackage{multirow}
\usepackage{amsmath}
\usepackage{geometry}
\geometry{letterpaper,tmargin=2.54cm,bmargin=2.54cm,lmargin=2.54cm,rmargin=2.54cm} 


\begin{document}

\title{Programming Assignment 2 - Population Structure}

\author{COMSCI/HUMGEN 124/224}

\date{Due: May 13, 2016, 11:59 pm}

\maketitle

This short programming assignment is designed to help you get an
understanding for the basics of population structure present in
human genetic data from a computational perspective. We have taken
the data from the 1000 Genomes Project, which sequenced a 30-60
individuals in about 20 populations all over the world. The populations
themselves are incredibly diverse, including poeple of Yoruba
descent in Nigeria, Han Chinese living in Beijing,
Gujarati Indians living in Houston, Italians in Tuscany,
Finns in Finland, and Americans of African and European descent.
I encourage you to explore \url{http://www.1000genomes.org} to
find out more about this incredibly important project in a very
active research area. The 1000 genomes were fully ascertained in 2015,
and projects involving 100,000 or more genomes (for example
Genomics England's \url{http://www.genomicsengland.co.uk}) are now underway.

You can use any language for this project,
though Python and R are recommended. You will
need to submit your code along with your results file through CCLE.


\section*{Computing the Kinship Matrix}

\subsection*{Reading the input}
The data for this programming assignment consists of a matrix
of 2,000 individuals and 1,000 SNPs. Unzip the file \verb!PopSNPs.zip!,
and you'll see a space-separated text file.
Each column of the input represents the number of copies of a single SNP, and each row
represents an individual. Each element in the matrix represents
the number of copies of a single SNP an individual has.

Your code for programming project 1 will come in handy here.
The data is also much more reasonably-sized, so the results
should generate much faster. If you've forgotten everything,
start with the documentation for \verb!pandas read_csv!,
or R's \texttt{read.table}. And don't forget to keep sharpening
your skills at using Googe to answer programming questions for you.


\subsection*{Part A - Standardizing Data}
The first step in this process will be to standardize the input data.
Standardization of data (sometimes also referred to as "normalization")
involves applying a linear function to the data so that it has mean 0
and variance 1.

There are libraries that will do this for you; look in Python \verb!scikit-learn!'s
\verb!preprocessing! module, and R has a builtin \verb!scale! method that
can be applied well here.

Standardizing the data isn't important for computing the kinship matrix
in the next step, but it's critically important for performing clustering

\subsection*{Part B - Kinship Matrix}
Our next job is to discover how closely related everyone
is to everyone else in our dataset.
\vspace{1.5 cm}

\subsection*{Part C - Clustering}

\subsection*{Part D - Visualization (required for grads, optional for undergrads)}

For grads, your final task is to visualize the clustering data you have made in part (C)
and post it to the forums. Some good examples of

\subsection*{Part E - Forums (required for undergrads, optional for grads)}



\clearpage
\subsection*{Part F - Output Format}
The most important part of this assignment is that you
input and output data in the proper format. The first line
should be your UID; it is recommended that you also put
your UID in the title of your file, but it's not required.

\begin{verbatim}
UID:{Your UID}
email:{Your email}
Undergrad or Grad:{Grad if you're a graduate student, undergrad otherwise}
<A>
{SNPNAME}:{RAW-P-VALUE}
{SNPNAME}:{RAW-P-VALUE}
{SNPNAME}:{RAW-P-VALUE}
...
</A>
<B>
{SIGNIFICANT-SNP1}
{SIGNIFICANT-SNP2}
...
</B>
<C>
Lambda_gc:<Lambda Value>
</C>
\end{verbatim}

If I were to submit an assignment, my output
would look something like this:

\begin{verbatim}
UID:123456789
email:bilow@cs.ucla.edu
Undergrad or Grad:Grad
<A>
SNP0000:0.175
SNP0001:0.875
SNP0002:0.0003
...
</A>
<B>
SNP0002
...
</B>
<C>
Lambda_gc:1.872
</C>
\end{verbatim}


\clearpage




\end{document}
